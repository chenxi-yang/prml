\documentclass[UTF8]{article}
\usepackage[a4paper,bindingoffset=0.2in,%
            left=2cm,right=2cm,top=1in,bottom=1in,%
            ]{geometry}
\usepackage[UTF8]{ctex}
\usepackage{lmodern}
\usepackage{graphicx}
\usepackage{subfigure}
\usepackage{ulem}
\usepackage{float}
\usepackage{enumitem}
\usepackage{listings}
\setlist{labelindent=\parindent}{}
\setlist[enumerate]{label=\textbf{\arabic*}}

\begin{document}

\begin{titlepage}
\title{\bfseries PRML SVM}
\author{杨晨曦 15307130360}
\maketitle
\tableofcontents
\thispagestyle{empty}
\end{titlepage}


\section{介绍}
本次作业用手写所有算法,没有使用sklearn之类的库。尽管性能和sklearn调用相比弱一些,但通过手写算法,更加确切的理解了算法的公式,加深了印象。作业分为三个大题,首先用kernel(线性,多项式和高斯)的SVM解决非线性可分的二分类问题\ref{kernel}。接下来,

\section{核函数处理非线性可分的二分类问题}\label{kernel}
核函数通常被用于kernel trick, $$k(\textbf)$$
\subsection{线性kernel}
\subsection{多项式kernel}
\subsection{高斯kernel}

\section{线性二分类问题}

\subsection{Linear Classification (squared error)}
\subsection{Logistic Regression (cross entropy error)}
\subsection{SVM (hinge error)}

\section{多分类问题}
\subsection{1vsOther}
\subsection{1vs}







\section{实验场景}
本次试验使用一台电脑,共计4个虚拟机完成。其中三台模拟分布式系统,一台为单机环境,将hive在这两个环境中的查询性能进行对比。
\subsection{分布式环境}\label{hadoop1}
\subsubsection{系统环境}
\begin{itemize}
	\item[] Ubuntu 14.04 
	\item[] hadoop 2.7.5 
	\item[] java 1.8.0\_161 
	\item[] hive 2.3.3
	\item[] 
	\item[] master: 10.211.55.25
	\item[] slave1: 10.211.55.27
	\item[] slave2: 10.211.55.28
\end{itemize}

\subsubsection{必要准备}

1. 添加hadoop用户,并且添加到sudoers(图~\ref{adduser}, \ref{user-pri})


\begin{figure*}[htb]\centering
\begin{minipage}[t]{0.48\textwidth}
\centering
\includegraphics[width=6cm]{figure/sshslave1.png}
\caption{ssh slave1}\label{ssh-slave1}
\end{minipage}
\begin{minipage}[t]{0.48\textwidth}
\centering
\includegraphics[width=6cm]{figure/sshslave2.png}
\caption{ssh slave2}\label{ssh-slave2}
\end{minipage}
\end{figure*}

\subsubsection{构建hadoop集群}

\begin{figure}[htb]\centering
	\includegraphics[width=\linewidth]{figure/core-site.png}
	\caption{core-site}\label{core-site}
\end{figure}
\begin{figure}[htb]\centering
	\includegraphics[width=\linewidth]{figure/hdfs-site.png}
	\caption{hdfs-site}\label{hdfs-site}
\end{figure}
\begin{figure}[htb]\centering
	\includegraphics[width=\linewidth]{figure/mapred-site.png}
	\caption{mapred-site}\label{mapred-site}
\end{figure}
\begin{figure}[htb]\centering
	\includegraphics[width=\linewidth]{figure/yarn-site.png}
	\caption{yarn-site}\label{yarn-site}
\end{figure}



\begin{figure}[htb] \centering
	\subfigure[伪分布式]{\label{local-6}
		\includegraphics[width=\linewidth]{figure/local20.png}
	}
	\subfigure[分布式]{\label{master-6}
		\includegraphics[width=\linewidth]{figure/master20.png}
	}
	\caption{实例5}\label{ins6}
\end{figure}


\section{性能比较}
五个实例在分布式和单机(伪分布式)中的运行时间,以此来反应性能。表~\ref{per}
\begin{table*}[!t]\centering
	\begin{tabular}{c | c c }
		\hline
		实例序号 & 分布式运行时间 /s & 单机环境运行时间 /s\\
		\hline
		1& 132.578 & 95.105\\
		2& 409.967 & 270.821\\
		3& 242.664 & 117.713\\
		4& 204.453 & 49.761\\
		5& 434.114 & 131.374\\
		\hline
	\end{tabular}
	\caption{Performance}\label{per}
\end{table*}



\section{总结}
搭了一个工作量挺大的环境,踩了不知道多少坑,写pj这段时间一直抱着csdn和stackoverflow,真的是救了命了。cs真的是一个大家不断分享的学科。也再一次感受到做cs相对底层研究的人的厉害和不容易。这段时间解决了很多问题,当一整天都在配hadoop的时候,解决一个问题又有一个问题真的挺崩溃的。但最终,还是一个人把整个系统搭完了,还是挺有成就感的!



\end{document}